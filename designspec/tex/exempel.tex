Detta avsnitt utgör ett exempel på användarinmatning och vad som händer bakom kulisserna till följd av denna.

\begin{enumerate}
  \item Användaren står i sökfönstervyn (fig. \ref{fig:vy1}) och klickar på \emph{``Lägg till''.}
  \item Vyn ändras till redigeringsvyn (fig. \ref{fig:vy3}). 
  \item Användarinmatning av receptdetaljer i respektive fält. Avslutar med \emph{``Spara recept''.}
  \item Skapa tomt \Recipe-objekt och anropa dess \verb=Recipe:editRecipe=-funktion med fältenas värden som argument.
  \item Ett anrop \verb=Shell::addRecipe(Recipe)= görs.
  \item Anrop: \verb=EditDB::addRecipe(Recipe)=
  \item \verb=EditDB= använder receptets namn och slår i databasen. Om det redan existerar skrivs informationen över. Om inte så läggs raden till i tabellen med recept.
  \item Returvärde returneras beroende av utfall.
  \item Nivå \Shell: \verb=currentRecipe_= uppdateras, returvärdet ovan returneras uppåt.
  \item Nivå GUI: GUI ger feedback baserat på returvärde (recept \{tillagt,uppdaterat\}).
\item Byt vy till receptvy (fig. \ref{fig:vy2}).
\end{enumerate}

