Programmet MatLabb har i stort tre delsystem:

\begin{enumerate}
\item \verb+GUI+, Det grafiska användargränssnittet
\item \verb+Shell+, eller, det inre skalet som håller centrala objekt och variabler, t.ex. aktivt recept och portionskalning.
\item \verb+Lookup+, som tillhandahåller relevanta verktyg för kommunikation med databasen.
\end{enumerate}
\verb+GUI+ ansvarar för att skriva ut data från det inre systemet på skärmen i grafisk tappning. Användaren styr även systemet genom att interagera med menyer, knappar och strängfält snarare än att ge skriftliga hänvisningar till kommandoprompten. Information presenteras enligt konceptbilderna i (figur \ref{fig:vy1} - \ref{fig:vy4}). Det grafiska gränssnittet implementeras med hjälp av biblioteket Qt och kommer delvis designas i klienten Qt Creator.

Shell innehåller objekt av klasserna \verb+Recipe+, \verb+IngredientInfo+ och \verb+Lookup+ som datamedlemmar, där de två förstnämnda reflekterar det aktuella receptet och ingrediensen som vårt program interagerar med. Utåt tillhandahåller \verb+Shell+ publikt endast funktioner som kan tänkas motsvara alla möjliga handlingar från användaren. Denna grupp av funktioner inkluderar, men är ej begränsade till, funktionerna listade i figur \ref{fig:tekfunklist} (s. \pageref{fig:tekfunklist}).

\begin{figure}[h]
\caption{Funktioner för användarinputs}
\begin{tabular}{p{5.5cm}|p{8cm}}
\verb+addRecipe(string+) & Konstruerar ett nytt tomt \verb+Recipe+-objekt för datamedlemmen \verb+currentRecipe_+. Ett defaultargument av datatypen string existerar för att potentiellt tilldela ett namn. \\[1.2mm]
\verb+editRecipe()+ & Kallar på \verb+currentRecipe_.editRecipe()+ för att kunna ändra på dess datamedlemmar.\\[1.2mm]
\verb+importTxt(string)+ & Importering från textfil. Konstruerar ett nytt \verb+Recipe+-objekt och försöker fylla i dess datamedlemmar enligt en standardmodell.\\[1.2mm]
\verb+exportTxt()+ & Kallar på \verb+currentRecipe_.exportTxt(string)+ för att exportera till .txt. Kan modifieras för att först hämta ett annat recept från databasen för exportering.\\[1.2mm]
\verb+addIngredient(string)+ & Motsvarande \verb+addRecipe+. \\[1.2mm]
\verb+editIngredient(string)+ &  Motsvarande \verb+addIngredient+. \\[1.2mm]
\verb++&\\[1.2mm]
\verb+matchRecipe(string)+ & Levererar en sträng till \verb+Lookup+-objektet för att slå i databasen för exakt matchning. \\[1.2mm]
\verb+matchIngredient(string)+ &  Motsvarande \verb+matchRecipe+. \\[1.2mm]
\verb+searchRecipe(cont<SearchTerm*>)+ & Levererar söktermer i en godtycklig container till \verb+Lookup+. \verb+recipeSearchResults_+ tilldelas det resultat som \verb+Lookup+ ger.  \\[1.2mm]
\verb++&\\[1.2mm]
\verb++& get-funktioner som används av GUI:t och returnerar relevant data. 
\end{tabular}
\label{fig:tekfunklist}
\end{figure}

Objekt av klasserna \verb+Recipe+ och \verb+IngredientInfo+ kommer endast att existera i stabilt tillstånd som datamedlemmar i klassen \verb+Shell+. Nya objekt skapas antingen i samband med t.ex. funktionen \verb+addRecipe(string)+ eller byggs upp och returneras av \verb+Lookup+ som resultat av en matchning i databasen. Klasserna \verb+Recept+ och \verb+IngredientInfo+ innehåller främst fullständig data om ett specifikt recept/ingrediens, men även funktioner för implementeringen av \verb+Shell+s ``användarfunktioner'', t.ex. åtkomst och redigering.

Endast ett objekt av klassen \verb+Lookup+ existerar i programmet och då som datamedlem av \verb+Shell+. \verb+Lookup+ ska inte hålla koll på några värden, utan fyller istället uppgiften att avgränsa funktionalitet och samla funktioner som har att göra med databaskommunikation på ett ställe.

 // Lookupfunktioner?



    

