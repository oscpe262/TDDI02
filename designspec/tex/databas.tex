Då all receptinforamtion behöver sparas mellan körningar av programmet
behöver den lagras externy. I det här programmet kommer en databas
användas med hjälp av MySQL. Databasen kan överskådas i
EER-diagramet i \ref{fig:erdiagram}. Den del som ligger innom de streckade området hör
till funktionalitet som endast kommer implementeras i mån av tid.

\begin{figure}[H]
        \centering
        \includegraphics[scale=0.35]{erdiagram.png}
        \caption{Entity-Relationship diagram}
        \label{fig:erdiagram}
\end{figure}

Databasen kommer bestå av fem entiteter \verb+Recipie+ som innehåller
information unik för ett enskilt recept. \verb+Ingridient+ som är en lista
över de olika ingredienser som databasen innehåller, \verb+Allergy+
som är en lista över allergier, \verb+Tool+ som är en lista över allergier samt
\verb+Comment+ som är en lista över kommentarer till varje recept

Entiteten \verb+recipie+ är den entiteten som lagrar namn, beskrivning, bild
tillagningsmetod, då recept skall ha unika namn för att särskilja dem åt är det
receptets namn som agerar primärnyckel.

\verb+recipie+ har en m-n relation till\verb+ingredient+ vilket ger oss en
lista på ingredienser till vajre recept. Genom att ha attributen \verb+kcal+ och \verb+price+
på entiteten \verb+ingredients+ istället för \verb+recipie+ behöver unik information om
portionspris och närings-inehåll till varje recept inte sparas utan kan räknas ut beroende på vilka
ingredienser som ingår. Genom att tillföra attributet \verb+amount+ behöver varje ingrediens
endast lagras en gång per recept och enhetsomvandling och portionsskalning kommer vara möjlig.

För att hålla ordning på de vanligaste matallergierna(samt kött mejeri och fisk för
veganer/veganer) finns entiteten \verb+Allergy+. Den har även en 1-n relation till \verb+Comment+
vilket resulterar i att alla recept får en lista med komentarer skrivna av användaren. samt en
m-n relation till \verb+Tool+ som ger en lista över de redskaps som behövs.

Genom att utforma databasen enligt sagda model kommer programmet kunna utföra sökningar olika
sökningar och filtreringar på ingredienser som ingår, inte ingår eventuella allergener etc.



