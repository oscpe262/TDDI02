\section{Matteus Laurent: Ledning och koordinering i ett projekts tidiga skede}

Innan utformandet av en designspecifikation kommer flera viktiga steg under ett projekts gång. Utgående från en väldigt grundläggande idé om en produkt, så ska idéer diskuteras och funktionalitet sammanställas för en kravspecifikation. Detta dokument lägger grunden för diskussionen kring designspecen. Det är i dessa tidiga skeden som projektet i stort tar form, och det är viktigt att alla ansvariga hänger med och kan bidra med konstruktiva åsikter för att fullborda en väl genomtänkt design. Detta leder till en mer realistisk implementering med färre frågetecken och effektivare gruppkommunikation - som förhoppningvis leder till att många timmar kan sparas in.

Som ledare för vår projektgrupp kan man anse att min roll är kritisk för att säkerställa det ovannämnda. Sålunda, en erfarenhet jag kommer att ta med mig från detta arbete är mitt agerande i denna roll, vad för påverkan det hade på dagordningen, och några förslag på korrigeringar för framtida projekt.

Några insatser jag inte är nöjda med:
\begin{itemize}
\item Jag försökte egga brainstorming utan att få först få alla ordentligt inblandade.
\item Jag var rädd för att styra och ställa för mycket, vilket ledde till:
\item Jag pressade inte individer tillräckligt för input.
\item Jag var ibland kritisk utan att kunna sätta fingret på varför.
\end{itemize}
Detta ledde till en skillnad i engagemang under designdiskussionen. De uttalande fick mest utrymme
och det fanns en tvekan till att framföra åsikter innan folk hade nått samma nivå.
Diskussionen hann inte få tid att mogna. Detta rörde till och ödslade tid när
designdokumentet skulle skrivas.

Utöver att ha ovannämnda i åtanke, så är min egna åtgärd att försöka dela upp diskussion i ytterligare pass,
så det finns tid att smälta första passet och fundera kring frågeställningarna. Kanske även ge i hemläxa
att förbereda programpunkter, för att få alla mer involverade från första början.
