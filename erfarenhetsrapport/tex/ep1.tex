\section {Erik Peyronson: Att sätta sig in i ett externt bibliotek}

I vårat projekt använder vi två externa bibliotek. QT samt MySql och
jag som databas-ansvarig har varit tvungen att sätta mig in i
båda vilket inte var helt lätt. När vi skulle designa vårat system
började jag försöka läsa på både om MySQL som jag visserligen hade
lite erfarenhet av sen tidigare men framförallt QT och det var oerhört
svårt att veta vilken ända man skulle börja i. Varje funktion eller
klass man stötte på var kopplad till en annan vilket ledde till långa
sidospår och många saker man behövde förstå för att förstå. 

Den stora miss jag gjorde var att inte direkt börja skriva ett litet
testprogram och testa de olika funktionerna som jag stötte på istället
för att bara försöka läsa sig till kunskapen, dels för att lättare
förstå hur dom fungerar och interagerar med varandra  men också för
att undvika missar i designen. 

Initialt känndes det som att det skulle vara ett slöseri med tid att
börja koda så tidigt i projektet då ingen användbar kod skulle
produceras men proceduren att få allting att fungera med kompilering,
inkluderingar etc var något som jag ändå var tvungen att göra när
kodningen väl satte igång och att testa mer hade definitivt hjälp mig
att spara tid i slutändan.
