\section{Johan Levinsson: Designplanering}

En upplevelse jag har haft i Programmeringsprojektet så har jag haft en upplevelse kring att planera designen. Jag har upplevt stor förvirring kring att planera designspecen. Jag fann modulerna i diagramen att vara högst vaga till en början, man var nästan tvungen till att dyka rakt in i det utan en klar vision och förståelse för att faktiskt få gjort någonting av substans, och först efteråt  försöka förstå sig på det på en djupare nivå.
När vi satt ner och ritade modulerna, så blev det till en början väldigt många. Vi har i TSIU03 gjort en liknande grej. Där går varje modul att konstruera i verkligheten, plocka ut, ta och titta på. I ett programmeringsprojekt på såpass hög nivå så är det mer vagt. Det fanns inte riktigt en tydlig definition om vad som skall vara en modul och vad som inte får vara en modul. Jag antar att det bär med sig att väldigt få saker inte "får" vara moduler. Denna frihet skulle, som mycket annat med få saker som är "fel", lämna många möjligheter för hur slutprodukten skall se ut. Då blir det förmodligen även svårt att visualisera hur det skall se ut när man är klar med det. 
Jag gillar inte hur vingligt den här approachen har känts för mig. Jag har som förhoppning att detta förvirrande moment kommer att gardera mig inför framtida liknande situationer, så att även om jag kanske inte fann det så bekvämt nu så kan det iallafall ge mig någon form av färdighet att ta med mig. Om det nu blir så eller inte återstår helt enkelt att se när projektet börjar närma sig klart. 
