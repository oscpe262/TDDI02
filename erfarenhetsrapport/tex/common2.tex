Datatyper och returtyper i designspecifikationen

Vid utformandet och skrivandet av en designspecifikation krävs beslutstagande i ett skede där detaljer ännu ej har stakats ut.

Man har en idé i stort om hur programmet ska se ut, eller snarare, hur det \emph{skulle} kunna se ut och hur det \emph{skulle} kunna implementeras. Det existerar många bra tankar och förslag, men just därför är det viktigt att som grupp bestämma sig för vilka lösningar man vill använda och på vilket sätt man vill implementera dem.

Väntar man, som vi gjorde, kan det senare under kodningsfasen komma upp oklarheter och egna tolkningar. Drjösmål kan uppstå för att man är osäker på vilket format man får information som ska användas för den modul man ansvarar för.

Ser man till att reflektera senare designändringar i specifikationen, behövs antagligen mer ineffektivt spenderad tid i det stadiet med en undermålig specifikation.

I vårt fall har vi missat på denna punkt. Många returtyper och parametertyper blev lämnade åt "senare". Tanken var att tillåta bättre idéer växa fram och undvika att låsa in sig, men resultatet av denna metod har istället haft mer negativa än positiva effekter på framfarten av vårt arbete.