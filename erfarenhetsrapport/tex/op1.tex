\section{Att introducera Git}

Valet att använda Git och Github för central lagring och versionshantering togs på ett tidigt stadie då det är ett populära och kraftfulla verktyg i sammanhanget. Givetvis innebar detta även en instegströskel extra då jag var den enda som i någon utsträckning använt mig utav det tidigare, och då endast i enkelanvändarmiljö.

Inledningsvis såg jag till att alla fixade användarkonton och därefter bifogades några länkar till vidare läsning samt en kortare hemskriven introduktion. Jag är dock tveksam till huruvida endera har lästs av någon involverad i projektet, och tack och lov gavs en översiktlig introduktionsföreläsning till Git relativt tidigt vilket gett gruppen ett hum om hur det fungerar.

Tack vare att mycket av projektets initiala del bestod av dokumentation (huvudsakligen via \LaTeX) så blev inkörningsperioden (där användarna bekantade sig med aktuella kommandon) tämligen överkomlig. Därmed dock inte sagt att det har gått smärtfritt att jobba med det. Git är ett kraftfullt och användarvänligt verktyg, men i likhet med det mesta som Torvalds haft fingrarna på så är det selektivt vilka användare det är vänligt emot. Missförstånd kring exempelvis hur commits fungerar har lett till ett par klassiska tillfällen där man valt att spara undan aktuellt arbete och radera lokala mappar för att därefter ``pulla'' hem repot på nytt. Det är givetvis inte ``rätt'' sätt att lösa problemen på, men det är åtminstone en lösning som sparat tid när det kommer till lösandet och fokus har då kunnat läggas på hur problemet undviks och på att jobba vidare på projektet.

Git är dock ett verktyg väl värt att bekanta sig med, och jag tror att vi alla kommer att vara glada över att ha gjort det förr eller senare.
