\section{Regelbundna möten}
Även om vi under hela första perioden haft frekventa gemensamma arbetspass så kände vi att det, när projektet nu gått in i det stadie där enskilt arbete blivit en större faktor, är hög tid att ha någon form av regelbundet koordinationsmöte för att underlätta distribuering av arbetsuppgifter, samt för att säkerställa att alla är med på banan och att alla vet vad som väntar och förväntas den närmaste tiden.

Vi bestämde oss därför för att i samband med handledarmötet varje vecka ha ett separat möte inom gruppen. Någon spikad dagordning har i skrivande stund inte upprättats, men de punkter som cementerades vid första mötet var:
\begin{itemize}
\item Vad har varje medlem gjort under den gångna veckan?
\item Vad ämnar varje medlem göra under den kommande veckan?
\item Arbetsmetodik: Vad fungerar bra, vad kan förbättras?
\item Stundande deadlines, delmål, planerad frånvaro etc.
\item Avstämning mot kravspec.
\end{itemize}

Att ha hållt dessa möten redan från start hade möjligen kunnat öka effektiviteten en aning, men då mycket av arbetet på det stadiet skedde gemensamt och gruppen även var en delmängd av den parallella projektkursens grupp så skedde en hel del av denna kommunikation fortlöpande.
