\section{Handledarmöten}
Under projektets gång så har vi upplevt en del dalar i arbetsgången. I början fanns en tid då vi inte gjorde några framsteg av substans. De möten som hölls varje vecka var trots detta obligatoriska, och vi gick därmed på dem. Under utbildningens gång har liknande projekt hållits utan obligatoriska möten. 

En positiv effekt som har kommit utav att gå på möten där allt som sägs är att vi inte har jobbat tillräckligt är att man får projektet färskt i minnet. Mötena i början agerade lite som en sporre. Man konstaterade att man inte hade arbetat nog, det var dags att börja arbeta. 

Snarare än den negativa utvärdering som skedde veckovis i början, så upplever vi att regelbundenheten satte starkare spår. Efter att man konstaterat att arbetet inte är tillräckligt för veckan man precis gått igenom så sår man i sinnet att nästa, obligatoriska möte så kommer i princip samma frågeställningar ställas. Det är inte ett alternativ att strunta i att jobba för att sedan strunta i att gå på nästa möte och slippa bemöta problemen. Valet står mellan att arbeta för dåligt, och sedan sitta och skämmas på mötet, eller att arbeta bra, och få en viss bekräftelse under mötets gång. 

Regelbundenheten, och att oavsett hur det gått, hålla möten, finner vi ha gett en positiv effekt. Detta är någonting som vi kommer att ta med oss i framtiden och tillämpa i framtida projekt. 
