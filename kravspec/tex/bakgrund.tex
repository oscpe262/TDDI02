Projektets syfte är att skapa en interaktiv receptdatabas med grafiskt användargränssnitt för att hjälpa användaren med organisering av recept med smarta sökfunktioner och för att underlätta matlagningsprocessen genom att tillåta portions- och enhetsomvandling. För att hjälpa användaren att hålla koll på både sin vikt och ekonomi kommer recepten även innehålla information om näringsinnehåll och pris.

MatLabb måste erbjuda något utöver det som står att finna i normala kokböcker. Att hitta recept i en kokbok kan vara klurigt eftersom att man begränsas till kokbokens index. Man kan exempelvis slå upp 'köttfärs' och få en lista med köttfärsbaserade rätter, men då finner man även rätter som innehåller ingredienser som man inte har tillgång till, rätter som tar längre tid än man har och rätter som på andra sätt faller utanför aktuella ramar. Därför vill vi kunna söka och finna recept på ett mer dynamiskt sätt.
