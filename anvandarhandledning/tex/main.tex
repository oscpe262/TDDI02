\em Många dokument riktar sig till utvecklare, en sån som du själv. Även om det förstås är viktigt att vara klar och tydlig, fri från motsägelser, och välstrukturerad när man skriver ett sånt har man i alla fall den fördelen att rikta sig till personer med samma bakgrund, och man delar fackkunskap, erfarenheter, terminologi och rent av 'jargong'. Rent språkligt är det därför inte så problematiskt att författa mer 'tekniska' dokument.

När det gäller en manual - användarhandledning - är situationen givetvis en helt annan. Man riktar sig nu till en 'anonym' läsekrets, om vars bakgrund och erfarenheter man inte vet något. Där kan finnas fullständiga noviser, likaväl som mycket väl insatta personer. Detta gör att det är mycket viktigt att lägga sig på 'rätt nivå' både vad gäller språkbruk och 'pedagogik'. Man får vare sig överskatta eller underskatta läsarna! Inte förutsätta att termer och handgrepp är välkända, men inte heller 'för barnsliga'. Inte använda vardagsslarvigt språk, och inte heller 'kanslispråk'. Det här är mycket kinkigt - d.v.s. både viktigt och knepigt!

En användarhandledning är ett dokument som man ska kunna läsa utan att ha programvaran till hands, men ändå i stora drag förstå hur man ska gå tillväga. Det måste ha en god pedagogisk struktur, så att 'kunskaper' och förståelse bygger på varandra. Specifika termer, som man gott kan använda, måste naturligtvis förklaras. Illustrationer är vanligen till god hjälp.

Strukturen på manualen skulle kunna vara denna:

\begin{itemize}Syftet med programvaran - vilket problem/vilken uppgift understöder den?
\item Allmän översikt över handhavandet, t.ex. generella principer för interaktionerna.
\item Förteckning över kommandon eller motsvarande: mina handgrepp, systemets responser. Listan kan kanske ha två nivåer - först grundläggande saker, sen lite mer avancerade kommandon, om sådan nivåindelning kan urskiljas.
\item Eventuellt kan man tillfoga en 'how to get started' / 'tutorial', om systemet är någorlunda omfattande. Alltså, en följd av inledande 'åtgärder' i logisk följd, och ganska detaljerat beskrivna, för att lösa ett typiskt problem.
\item En avslutande koncis sammanfattning, en 'snabbreferens' över kommandon (eller motsvarande) kan behövas. Speciellt om det finns många kommandon.
\end{itemize}
\em

%%%%%%%%%%%%%%%%%%%%%%%%

\input{./tex/foo}
