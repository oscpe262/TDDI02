När Matlabb startas möts man av Matlabbs hjärta, Sök-vyn. Det är här
användaren söker efter och öppnar recept. Den enkla användaren behöver
bara trycka på \verb+sök+knappen för att få upp en lista över samtliga
recept och kan enkelt öppna det recept den är intresserad av av
recepten som dyker upp i listan. Men det finns även två sätt att söka
efter recept i matlabb, Titelsök och Filtersök.

\begin{figure}[h]
        \centering 
        \includegraphics[scale=0.44]{sok_recept.png} 
        \caption{Sökvyn} 
        \label{fig:sokvyn}
\end{figure}


\subsection{Titelsökning}
Vet man redan vilket recept man är ute efter kan det enkelt och snabbt
hittas med hjälp av titelsökfunktionen. Receptets namn matas in i
titelsök rutan och hittas med \verb+titelsök+-knappen och kan sedan öppnas
med knappen \verb+Öppna+. Medans detta räcker för många kan även den
avancerade användaren istället filtrera sina sökningar.

\subsection{Filtersökning}

Matlabb har ett flertal avancerade filtreringsmekanismer, varav den
mest centrala är Ingrediensfiltrering. Ingrediensfiltret finns till
vänster i sök-vyn märkt Ingredienser. För att inkludera en ingrediens
i listan markeras den och flyttas till Sök-listan med hjälp av \verb+>>+
knappen. Vill man ta bort en ingrediens från sök-listan görs detta med
hjälp av \verb+<<+ knappen.

Till höger om Ingreienslistan ses två liknande listor. Allergier samt
kosthållning som används för att filtrera bort recept innehållande en
viss allergen eller ifall man vill exkludera recept av etiska eller
religiösa skäl. Dessa fungerar på precis samma sätt som för
ingrediensfiltrering.

Under kosthållningslistan finns filtren för Energiinnehåll tidsåtgång
och portionskostnad. Filtreringen görs genom att med hjälp av pilarna
ange ett intervall om kommer begränsa sökningen till recept innom
intervallet.

När filtreringsinställningarna är färdiga slutförs sökningen genom att
knappen \verb+Sök+ trycks in och recepten dyker upp i listan till
höger. För att öppna ett recept markeras dess namn och när
knappen \verb+öppna+ trycks in skickas man vidare till recept-vyn.
