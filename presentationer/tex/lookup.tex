%Slide 1 inledning

\begin{frame}
  \frametitle{Lookup}
  \framesubtitle{Inledning}
  \begin{itemize}
    \item<1-> Klassen ansvarig för databasuppslagningar
    \item<2-> Datamedlem för senaste databassökning, senaste ingrediensobjekt samt position
    \item<3-> Grundläggande sökfunktioner
    \item<4-> Funktioner för dataåtkomst
    \item<5-> Funktioner för sammanslagningar av sökresultat
 \end{itemize}
\end{frame}

%Slide sökfunktioner

\begin{frame}
  \frametitle{Lookup}
  \framesubtitle{Sökfunktioner}
  \begin{itemize}
    \item<1-> \texttt{query\_list()}
    \item<2-> \texttt{query\_ingredient\_list()}
    \item<3-> \texttt{query\_ingredient\_list\_explicit()}
    \item<4-> \texttt{query\_allergy\_list()}
    \item<5-> \texttt{query\_price\_list()}
    \item<6-> \texttt{query\_calory\_list()}
  \end{itemize}
\end{frame}

%Slide 3 Sammanslagningsfunktioner
\begin{frame}
  \frametitle{Lookup}
  \framesubtitle{Sammanslagningsfunktioner}
  \begin{itemize}
    \item<1-> \texttt{union()}
    \item<2-> \texttt{intersect()}
    \item<3-> \texttt{complement()}
  \end{itemize}
\end{frame}

%Slide 4 Dataåtkomst
\begin{frame}
  \frametitle{Lookup}
  \framesubtitle{Funktioner för dataåtkomst}
  \begin{itemize}
    \item<1-> \texttt{get\_list()}
    \item<2-> \texttt{get\_recipe()}
    \item<3-> \texttt{get\_info\_ingredient()}
    \item<4-> \texttt{get\_recipe\_ingredient()}
  \end{itemize}
\end{frame}


