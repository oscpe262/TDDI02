%%%%%%%%%%%%%%%%%%%%%%%%%%%%%%%%%%%%%%%%%%%%%%%%%%%%%%%%%%%%%%%%%%%%%%%%%%%%%%%%%%%%%%%%%%
% Dokumenttypen Beamer låter oss skapa slides m.h.a. LaTeX
%
% I stort sett allt fungerar som ni är vana vid när det kommer till  formatering, men
% en markant skillnad är \verb som inte fungerar.  Använd i stället
% \texttt{<insert text here>}.
%
% Ny slide sätts i miljön ``frame''. Se gärna dummies och sliden i TSIU03 för basic stuff.
% Full dokumentation återfinns på:
% http://ftp.acc.umu.se/mirror/CTAN/macros/latex/contrib/beamer/doc/beameruserguide.pdf
%%%%%%%%%%%%%%%%%%%%%%%%%%%%%%%%%%%%%%%%%%%%%%%%%%%%%%%%%%%%%%%%%%%%%%%%%%%%%%%%%%%%%%%%%%

\begin{frame}
  \frametitle{MatLabb}
  \framesubtitle{Problemställning}
  \begin{itemize}
    \item<2-> Kända ingredienser -- vad gör vi av dessa?
    \item<3-> Förslag på rätter med begränsningar (vegetariskt, allergier, etc.)
    \item<4-> Taskig ekonomi? 
    \item<5-> Räknar kalorier?
  \end{itemize}
\end{frame}

\begin{frame}
  \frametitle{MatLabb}
  \framesubtitle{Vad är MatLabb?}
  \begin{itemize}
    \item<1-> Interaktiv kokbok
    \item<2-> Organisera recept
    \item<3-> Söka recept baserat på exempelvis:
      \begin{itemize}
        \item Receptnamn
        \item Ingredienser
        \item Allergier
        \item Kostpreferenser (Ex. vegetarisk)
      \end{itemize}
    \item<4-> Underlätta portions- och enhetsomvandling
    \item<5-> Uppskattat energiinnehåll
  \end{itemize}
\end{frame}
