%%%%%%%%%%%%%%%%%%%%%%%%%%%%%%%%%%%%%%%%%%%%%%%%%%%%%%%%%%%%%%%%%%%%%%%%%%%%%%%%%%%%%%%%%%
% Dokumenttypen Beamer låter oss skapa slides m.h.a. LaTeX
%
% I stort sett allt fungerar som ni är vana vid när det kommer till  formatering, men
% en markant skillnad är \verb som inte fungerar.  Använd i stället
% \texttt{<insert text here>}.
%
% Ny slide sätts i miljön ``frame''. Se gärna dummies och sliden i TSIU03 för basic stuff.
% Full dokumentation återfinns på:
% http://ftp.acc.umu.se/mirror/CTAN/macros/latex/contrib/beamer/doc/beameruserguide.pdf
%%%%%%%%%%%%%%%%%%%%%%%%%%%%%%%%%%%%%%%%%%%%%%%%%%%%%%%%%%%%%%%%%%%%%%%%%%%%%%%%%%%%%%%%%%

\begin{frame}
  \frametitle{Shell}
  \framesubtitle{Datamedlemmar och funktioner, översikt}
  \begin{itemize}
    \item Objekt av \texttt{Recipe, InfoIngredient} och \texttt{Lookup} som datamedlemmar
    \item<2-> Tillhandahåller utåt de funktioner som motsvarar avsedd användarinput, exempelvis:
    \item<2-> \texttt{
      addRecipe(string), editRecipe(),
      addIngredient(string), editIngredient(),
      importTxt(string), exportTxt(),
      matchRecipe(string), matchIngredient(string),
      searchRecipe(cont<SearchTerm*>)
    }
  \end{itemize} 
\end{frame}

\begin{frame}
  \frametitle{Shell}
  \framesubtitle{Klass: Recipe}
  \begin{itemize}
  \item Ett \texttt{Recipe}-objekt innehåller all information om ett specifikt recept.
  \item \texttt{Shell}s datamedlem currentRecipe\_ motsvarar det aktiva sådant
    \begin{itemize}
    \item Används av exempelvis \texttt{editRecipe()} och \texttt{exportTxt()}
    \end{itemize}
  \item \texttt{MiniRecipe} är en nerbantad datatyp av \texttt{Recipe}
  \end{itemize}
\end{frame}

\begin{frame}
  \frametitle{Shell}
  \framesubtitle{Klass: Ingredient}
  \begin{itemize}
  \item Abstrakt klass med två arv: \texttt{InfoIngredient} och \texttt{RecipeIngredient}
    \begin{itemize}
    \item \texttt{InfoIngredient} visar data och kan redigeras
    \item \texttt{RecipeIngredient} används av \texttt{Recipe} och har kvantitet
    \end{itemize}
  \end{itemize}
\end{frame}

\begin{frame}
  \frametitle{Shell}
  \framesubtitle{Klass: Lookup}
  \begin{itemize}
  \item \texttt{Lookup} - modul och klass för kommunikation med databasen
  \item Datamedlem i \texttt{Shell}, \texttt{**searchEngine\_}
  \item Exempel på \texttt{Shell}-funktioner som använder \texttt{Lookup}:
    \begin{itemize}
    \item \texttt{matchRecipe(string)}
    \item \texttt{searchRecipe(cont<SearchTerm*>)}
    \item Resultat uppdaterar \texttt{Shell}s datamedlemmar
    \end{itemize}
  \end{itemize}
\end{frame}

% Ev. RelatedRecipe, är den essentiell nog att ta upp i presentationen?
