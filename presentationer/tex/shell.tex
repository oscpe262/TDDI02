%%%%%%%%%%%%%%%%%%%%%%%%%%%%%%%%%%%%%%%%%%%%%%%%%%%%%%%%%%%%%%%%%%%%%%%%%%%%%%%%%%%%%%%%%%
% Dokumenttypen Beamer låter oss skapa slides m.h.a. LaTeX
%
% I stort sett allt fungerar som ni är vana vid när det kommer till  formatering, men
% en markant skillnad är \verb som inte fungerar.  Använd i stället
% \texttt{<insert text here>}.
%
% Ny slide sätts i miljön ``frame''. Se gärna dummies och sliden i TSIU03 för basic stuff.
% Full dokumentation återfinns på:
% http://ftp.acc.umu.se/mirror/CTAN/macros/latex/contrib/beamer/doc/beameruserguide.pdf
%%%%%%%%%%%%%%%%%%%%%%%%%%%%%%%%%%%%%%%%%%%%%%%%%%%%%%%%%%%%%%%%%%%%%%%%%%%%%%%%%%%%%%%%%%

\begin{frame}
  \frametitle{Shell}
  \framesubtitle{Datamedlemmar och funktioner, översikt}
  \begin{itemize}
    \item Objekt av \texttt{Recipe, InfoIngredient} och \texttt{Lookup} som datamedlemmar
    \item<2-> Tillhandahåller utåt de funktioner som motsvarar avsedd användarinput, exempelvis:
    \item<2-> \texttt{
      addRecipe(string), editRecipe(),
      addIngredient(string), editIngredient(),
      importTxt(string), exportTxt(),
      matchRecipe(string), matchIngredient(string),
      searchRecipe(cont<SearchTerm*>)
    }
  \end{itemize} 
\end{frame}

\begin{frame}
  \frametitle{Shell}
  \framesubtitle{Objekt: Recipe}
\end{frame}

\begin{frame}
  \frametitle{Shell}
  \framesubtitle{Objekt: InfoIngredient} 
\end{frame}

% Ev. RelatedRecipe, är den essentiell nog att ta upp i presentationen?
